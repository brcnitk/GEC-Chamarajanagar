\documentclass[11pt]{article}
\usepackage[margin=4cm]{geometry}
% algorithmic environment using algpseudocode package
\usepackage{algpseudocode}
\usepackage{algorithm}

\begin{document}
  
\section{Examples algorithm or pseudocode style and statements using algorithemic environment by algpseudocode pakage}

\listofalgorithms

Here we demonstrate how algorithms or pseudocode can be typeset using the \verb|algorithmic| environment provided by the \verb|algpseudocode| package.



\subsection{ If-then-else example without caption}

Note that the command names provided by \verb|algpseudocode| are typically title-cased, e.g.~\verb|\State|, \verb|\While|, \verb|\EndWhile|.

% If you would like to add line numbers to the algorithm, you can add the first line number to the environment as an optional argument \begin{algorithmic}[1]
 
\begin{algorithmic}
\State $i \gets 10$
\If{$i\geq 5$} 
    \State $i \gets i-1$
\Else
    \If{$i\leq 3$}
        \State $i \gets i+2$
    \EndIf
\EndIf 
\end{algorithmic}

The above algorithm example is not captioned. If you need a captioned algorithm, load the \verb|algorithm| package, and add 
\subsection{ If-then-else example with caption}

\begin{algorithm}
\caption{An algorithm with caption}\label{alg:cap}
\begin{algorithmic}
\State $i \gets 10$
\If{$i\geq 5$} 
    \State $i \gets i-1$
\Else
    \If{$i\leq 3$}
        \State $i \gets i+2$
    \EndIf
\EndIf 
\end{algorithmic}
\end{algorithm}

around your \verb|algorithmic| environment. You can use \verb|\label{...}| after the \verb|\caption| so that the algorithm number can be cross-referenced, e.g.~Algorithm~\ref{alg:cap} and \ref{alg:wordy}.

\begin{algorithm}[hbt!]
\caption{Example algorithm for use of while condition with caption}\label{alg:wordy}
\begin{algorithmic}
\Require{Write here the required data}
\Ensure{Write here the expected result}
 \State initialization;
 \While{While condition}
  \State instructions;
  \If{condition}
   \State instructions1;
   \State instructions2;
  \Else
   \State instructions3;
  \EndIf
 \EndWhile
\end{algorithmic}
\end{algorithm}





\begin{algorithm}[hbt!]
\caption{An algorithm with caption}\label{alg:cap}
\begin{algorithmic}
\Require $n \geq 0$
\Ensure $y = x^n$
\State $y \gets 1$
\State $X \gets x$
\State $N \gets n$
\While{$N \neq 0$}
\If{$N$ is even}
    \State $X \gets X \times X$
    \State $N \gets \frac{N}{2} $  \Comment{This is a comment}
\ElsIf{$N$ is odd}
    \State $y \gets y \times X$
    \State $N \gets N - 1$
\EndIf
\EndWhile
\end{algorithmic}
\end{algorithm}

The \verb|algorithm| environment is a \emph{float}, like \verb|table| and \verb|figure|, so you can add float placement modifiers \verb|[hbt!]| after \verb|\begin{algorithm}| if necessary.



The \verb|algorithm| package also provides a \verb|\listofalgorithms| command that works like \verb|\listoffigures|, but for captioned algorithms:



\end{document}



