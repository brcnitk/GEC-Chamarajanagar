\documentclass[10pt]{article}
\usepackage[utf8]{inputenc}
\usepackage[T1]{fontenc}
\usepackage{amsmath}
\usepackage{amsfonts}
\usepackage{amssymb}
\usepackage[version=4]{mhchem}
\usepackage{stmaryrd}

\begin{document}

\section{ Trigonometric equations}

a) $\quad \sin x=-\frac{1}{2}$\\

b) $\quad \sin x=\frac{\sqrt{2}}{2}$\\

c) $\quad \tan x=\frac{\sqrt{3}}{3}$\\

d) $\quad \sin ^2 x=\frac{1}{2}$\\

e) $\quad \cos \left(\frac{\pi}{4}-x\right)=1$\\

f) $\quad \sin \left(\frac{\pi}{3}-x\right)=-\frac{1}{2}$\\

g) $2 \sin ^2 x=\sqrt{2} \sin x$\\

h) $\cot ^2 x=-\cot x$\\

i) $\quad \sin ^2 x-\cos ^2 x+\sin x=0$\\

j) $\quad 2 \tan x-3 \cot x=1$\\

k) $\quad 3 \tan ^2 x+4 \sqrt{3} \tan x+3=0$\\

l) $\quad \sqrt{3} \cot ^2 x-2 \cot x-\sqrt{3}=0$\\

m) $\frac{\sqrt{3}}{\sin ^2 x}+4 \cot x=0$\\

n) $\frac{\tan x+1}{\tan x-1}=2+\sqrt{3}$\\

o) $2 \cos ^2 x-7 \cos x+3=0$\\

p) $\frac{\sqrt{3}}{\cos ^2 x}-4 \tan x=0$\\

q) $2+\cos 2 x=-5 \sin x$\\

r) $\quad \sin (4 x-1)=0$\\

s) $\quad \sin x \cos x=\frac{1}{2}$\\

t) $\quad \sin ^2 x-\sin x=0$\\

u) $\quad 2 \cos ^2 x=\sin ^2 x-1$\\

v) $\quad \sin x+\cos x=\frac{1+\sqrt{3}}{2}$\\

w) $3 \tan x-1=2 \tan x$\\

x) $\quad \sin x+\sin 2 x=\sin 3 x$


\subsection{Geometric equations}

$$
\begin{aligned}
&\mathrm{S}_n=a_1+a_1 r+a_1 r^2+a_1 r^3+\cdots+a_1 r^{n-1} \\

&\mathrm{~S}_n=a_1 r+a_1 r^2+a_1 r^3+\cdots+a_1 r^{n-1}+a_1 r^n \\

&\mathrm{~S}_{\mathrm{n}}-\mathrm{rS}_{\mathrm{n}}=\mathrm{a}_1-\mathrm{a}_1 \mathrm{r}^{\mathrm{n}} \\

&\mathrm{S}_{\mathrm{n}}(1-\mathrm{r})=\mathrm{a}_1\left(1-\mathrm{r}^{\mathrm{n}}\right) \\

&\mathrm{S}_{\mathrm{n}}=\frac{\mathrm{a}_1\left(1-\mathrm{r}^{\mathrm{n}}\right)}{(1-\mathrm{r})}\\

\end{aligned}
$$


\subsection{Complex numbers}

\textbf{LR-C network}

$z=a+j b=r(\cos \theta+j \sin \theta)=r \angle \theta$

where $j^2=-1$ Modulus, $r=|z|=\sqrt{ }\left(a^2+b^2\right)$

Argument, $\theta=\arg z=\tan ^{-1} \frac{b}{a}$\\


\textbf{LR-CR network}


Addition: $(a+j b)+(c+j b)=(a+c)+j(b+d)$

Subtraction: $(a+j b)-(c+j d)=(a-c)+j(b-d)$

Complex equations: If $a+j b=c+j d$, then $a=c$ and
$f_r=\frac{1}{2 \pi \sqrt{ }(L C)} \sqrt{\left(\frac{R_L^2-L / C}{R_C^2-L / C}\right)}$
$b=d$
If $z_1=r_1 \angle \theta_1$ and $z_2=r_2 \angle \theta_2$ then


Determinants
Multiplication: $z_1 z_2=r_1 r_2 \angle\left(\theta_1+\theta_2\right)$

and Division: $\frac{z_1}{z_2}=\frac{r_1}{r_2} \angle\left(\theta_1-\theta_2\right)$\\

\textbf{De Moivre's theorem:}
$[r \angle \theta]^n=r^n \angle n \theta=r^n(\cos n \theta+j \sin n \theta)$
$$
\begin{aligned}
&\left|\begin{array}{ll}
a & b \\
c & d
\end{array}\right|=a d-b c \\
&\left|\begin{array}{lll}
a & b & c \\
d & e & f \\
g & h & j
\end{array}\right|=a\left|\begin{array}{ll}
e & f \\
h & j
\end{array}\right|-b\left|\begin{array}{ll}
d & f \\
g & j
\end{array}\right|+c\left|\begin{array}{ll}
d & e \\
g & h
\end{array}\right|
\end{aligned}
$$


\subsection{Fourier Series Equations}
$$
\begin{aligned}
&a_0=\frac{1}{\pi} \int_T^{T+2 \pi} f(x) d x \\
&a_n=\frac{1}{\pi} \int_T^{T+2 \pi} f(x) \cos (n x) d x \\
&b_n=\frac{1}{\pi} \int_T^{T+2 \pi} f(x) \sin (n x) d x
\end{aligned}
$$

\subsection{Examples of Mechanical engineering equations}

$$
\begin{aligned}
\sigma_{x^{\prime}} &=\frac{\sigma_x+\sigma_y}{2}+\frac{\sigma_x-\sigma_y}{2} \cos 2 \theta+\tau_{x y} \sin 2 \theta \\
\sigma_{y^{\prime}} &=\frac{\sigma_x+\sigma_y}{2}-\frac{\sigma_x-\sigma_y}{2} \cos 2 \theta-\tau_{x y} \sin 2 \theta \\
\tau_{x^{\prime} y^{\prime}} &=-\frac{\sigma_x-\sigma_y}{2} \sin 2 \theta+\tau_{x y} \cos 2 \theta
\end{aligned}
$$


$$
\begin{aligned}
&\sigma_{\mathrm{I}}=\sigma_{\mathrm{x}^{\prime}, \max }=\frac{\sigma_{\mathrm{x}}+\sigma_{\mathrm{y}}}{2}+\sqrt{\left(\frac{\sigma_{\mathrm{x}}-\sigma_{\mathrm{y}}}{2}\right)^2+\tau_{\mathrm{xy}}^2} \\
&\sigma_{\mathrm{II}}=\sigma_{\mathrm{x}^{\prime}, \min }=\frac{\sigma_{\mathrm{x}}+\sigma_{\mathrm{y}}}{2}-\sqrt{\left(\frac{\sigma_{\mathrm{x}}-\sigma_{\mathrm{y}}}{2}\right)^2+\tau_{\mathrm{xy}}^2}
\end{aligned}
$$

\newpage

\subsection{Examples of Civil Engineering Equations}
\textbf{Strength of Materials}

$$
\begin{aligned}
&\sigma=\frac{P}{A}=E \varepsilon \\
&\delta=\frac{P L}{A E} \\
&\delta_T=\alpha(\Delta T) L \\
&\sigma=\frac{P}{A_o} \cos ^2 \theta \\
&\tau=\frac{P}{A_o} \cos \theta \sin \theta \\
&\tau=\frac{P}{A} \\
&\sigma_b=\frac{P}{A}=\frac{P}{t d} \\
&F S .=\frac{P_{\text {ult }}}{P_{\text {all }}}=\frac{\sigma_{\text {ult }}}{\sigma_{\text {all }}} \\
&v=-\frac{\text { lateral strain }}{\text { axial strain }} \\
&\sigma_{\max }=K \frac{P}{A_{n e t}} \\
&\tau_{\max }=K \frac{T c}{J} \\
&\tau=\frac{\rho}{c} \tau_{\max } \\
&\tau=\frac{T \rho}{J} \\
&\varphi=\frac{T L}{J G} \\
&P=T \omega \\
&\omega=2 \pi f \\
&P=2 \pi T f \\
&e=\varepsilon_x+\varepsilon_y+\varepsilon_z \\
&
\end{aligned}
$$


$$
\begin{aligned}
&\sigma_{x^{\prime}}=\frac{\sigma_x+\sigma_y}{2}+\frac{\sigma_x-\sigma_y}{2} \cos 2 \theta+\tau_{x y} \sin 2 \theta \\
&\sigma_{y^{\prime}}=\frac{\sigma_x+\sigma_y}{2}-\frac{\sigma_x-\sigma_y}{2} \cos 2 \theta-\tau_{x y} \sin 2 \theta \\
&\tau_{x^{\prime} y^{\prime}}=-\frac{\sigma_x-\sigma_y}{2} \sin 2 \theta+\tau_{x y} \cos 2 \theta \\
&\sigma_{a v e}=\frac{\sigma_x+\sigma_y}{2} \\
&R=\sqrt{\left(\frac{\sigma_x-\sigma_y}{2}\right)^2+\tau_{x y}{ }^2} \\
&\tan 2 \theta_p=\frac{2 \tau_{x y}}{\sigma_x-\sigma_y} \\
&\sigma_{\max , \min }=\frac{\sigma_x+\sigma_y}{2} \pm \sqrt{\left(\frac{\sigma_x-\sigma_y}{2}\right)^2+\tau_{x y}^2} \\
&\tan 2 \theta_s=-\frac{\sigma_x-\sigma_y}{2 \tau_{x y}} \\
&\tau_{\max }=\sqrt{\left(\frac{\sigma_x+\sigma_y}{2}\right)^2+\tau_{x y}{ }^2} \\
&X\left(\sigma_x,-\tau_{x y}\right) \quad Y\left(\sigma_y, \tau_{x y}\right) \\
&\text { Cylinders } \quad \sigma_1=\frac{p r}{t} \quad \sigma_2=\frac{p r}{2 t} \quad \tau_{\max }=\frac{p r}{4 t} \\
&\text { Spheres } \sigma_1=\sigma_2=\frac{p r}{2 t} \quad \tau_{\max }=\frac{p r}{4 t} \\
&\varepsilon_{x^{\prime}}=\frac{\varepsilon_x+\varepsilon_y}{2}+\frac{\varepsilon_x-\varepsilon_y}{2} \cos 2 \theta+\frac{\gamma_{x y}}{2} \sin 2 \theta \\
&\varepsilon_{y^{\prime}}=\frac{\varepsilon_x+\varepsilon_y}{2}-\frac{\varepsilon_x-\varepsilon_y}{2} \cos 2 \theta-\frac{\gamma_{x y}}{2} \sin 2 \theta \\
&\gamma_{x^{\prime} y^{\prime}}=-\left(\varepsilon_x-\varepsilon_y\right) \sin 2 \theta+\frac{\gamma_{x y}}{2} \cos 2 \theta \\
&\varepsilon_{\text {ave }}=\frac{\varepsilon_x+\varepsilon_y}{2} \\
&R=\sqrt{\left(\frac{\varepsilon_x-\varepsilon_y}{2}\right)^2+\frac{\gamma_{x y}{ }^2}{2}} \\
&\tan 2 \theta_p=\frac{\gamma_{x y}}{\varepsilon_x-\varepsilon_y} \\
&
\end{aligned}
$$
\end{document}