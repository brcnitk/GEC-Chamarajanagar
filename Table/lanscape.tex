\documentclass{article}

% For \begin{longtblr} ... \end{longtblr}
\usepackage{tabularray}
\usepackage[table,xcdraw]{xcolor}

% For \begin{sidewaystable} .... \end{sidewaystable}
\usepackage{rotating}

% For \begin{landscape} ... \end{landscape}
\usepackage{lscape}

\title{Cartesian closed categories and the price of eggs}
\author{Jane Doe}
\date{September 1994}
\begin{document}
   \maketitle
   Hello world!
   
% \begin{sidewaystable}
\begin{landscape}
\begin{longtblr}[
    caption = {Existing algorithms issues and their consequences},
    label = {issues}
    ]{
    colspec = {|c|m{1.8in}|m{1.8in}|},
    rowhead = 1,
    hlines,
    % row{even} = {gray9},
    row{1} = {olive9},
    }
    \hline
    \textbf{Sl. No.} & \centering \textbf{Issues} & \centering \textbf{Consequences} \\ \hline
    1 & \centering Void node is a part of routing & \centering Higher packet loss \\ \hline
    2 & \centering Limited number of next hop selection attributes & \centering Inappropriate selection of next hop or cluster\\ \hline
    3 & \centering Duplicate retransmission & \centering Reduced lifetime of the node and network \\ \hline
    4 & \centering Not scalable & \centering Limited use in dense network \\ \hline
    5 & \centering Complex algorithm & \centering Higher communication overhead \\ \hline
    6 & \centering Overloading of some nodes as a cluster head & \centering Reduced lifetime of the network \\ \hline
\end{longtblr}
% \end{sidewaystable}
\end{landscape}

\end{document}
